Az adatbázisok használata a számítógéppel egyidős, az adatok rendszerezése, karbantartása, kereshetősége a korábbi papíralapú rendszereknél komoly problémát okozott. A számítógép és internet alapú világban az adatbázisok használata mindennapossá vált, a számítógépeken, telefonokon futó alkalmazások legnagyobb része használ valamilyen adatbázist. Eddig az adatbázis fogalma alatt mindenki egy relációs rendszerre gondolt, amely sok problémára megoldást nyújt, azonban legalább ennyire csak szuboptimális. A dolgozat olyan rendszereket mutat be, amelyekkel megoldható a vertikális és a horizontális skálázhatóság, a cachelés, a memória optimális- és maximális kihasználása, a geolokalizációs információk natív indexelése és keresése, miközben az adatsorok mérete, és az oszlopok száma rugalmas. Mindez SQL tudás nélkül, megőrizve a relációs rendszerek legtöbb előnyét.\\
\\
%Az egyetemi tanulmányaim mellett webfejleszőként dolgozom, így a relációs adatbázisok használata a mindennapos munkám része. A projektek fejlesztésekor az adatbázisok pozitív és negatív %oldalával is találkoztam már. Azonban folyamatosan foglalkoztat a kérdés, hogy hogyan lehetne rövidebb fejlesztési idővel, kevesebb programozással jobb minőséget, nagyobb rendelkezésre állást %és kisebb erőforrásfelhasználást elérni.\\
%A kérdésemre a dokumentum-orientált adatbázisok adták meg választ
\\
A dokumentum-orientált adatbázisokkal az alkalmazások fejlesztése nem csak felgyorsulhat, hanem olyan problémákra is megoldást nyújthat, melyek megoldása relációs rendszerekkel lehetetlen vagy nagyon nehéz. A rugalmas méretű adatsorok segítségével az adatbázis könnyedén felveszi az inkrementális fejlesztés támasztotta elvárásokat, és áthidalja a relációs adatbázisoknál gyakran felmerülő hézagos adatfeltöltés miatti átláthatatlanságot. \\
A memória optimális kihasználásával (akár egy teljes klaszter memóriáját képes kihasználni), az adatbázis-kezelő akár a relációs rendszerek válaszidejének töredéke alatt tudja kiszolgálni a kéréseket.\\
Az alapértelmezetten szállított vertikális skálázással (sharding\nomenclature{Sharding}{\hfill\\A vertikális skálázás egyik formája, ilyenkor egy adatbázis szerver mesterként funkcionál, bár adatokat nem tárol, és a mester irányítja a shardingbe bevont kliens szerveket, melyek önállóan működnek, azonban egyik kliensen sincs meg a teljes adatbázis, mindegyik csak az információ egy részét tartalmazza.\\A kliensek nem tudják, hogy ők shardingbe vannak kötve.}), egy-egy új szerver bekapcsolása a szolgáltatásba lerövidülhet, ezáltal csökkentve az emberi erőforrásigényt, és növelve a szolgáltatásminőséget és folytonosságot.

% dolgozatomban bemutatom, hogy a jelenleg használt relációs adatbázisok miért nem tudják kielégíteni egy új, modern alapvetően web2.0-ás szolgáltatás igényeit, és továbbá felvázolom Facebook, Twitter, Foursquare oldalaknál használt dokumentum-orientált megoldások előnyeit illetve hátrányait.
