Az adatbázisokat Gajdos Sándor a Budapesti Műszaki és Gazdaságtudományi Egyetem adjunktusa a következőképpen definiálja: \emph{adatok gyors, gépesített tárolása és visszakeresése} (\defcitealias{gajdos}{Gajdos, 2006, 4. old}\citetalias{gajdos}).\\
Az adatbázisok használatára már az elektromechanikus számológépek idejében felmerült az igény, igaz ekkor még csak lyukkártyára lyukasztották az információkat. Az adatbázisok térnyerése az 1960-as évekre tehető, amikor a számítógépek kizárólagos katonai szerepe megszűnt, mivel azok megfizethetővé váltak a cégek számára. Az ekkor használt rendszerek még az adatokat nem egy struktúraként kezelték, hanem az egyes rekordokra helyezték a hangsúlyt.
\\
Az igazi áttörést, az IBM által  az 1970-es években létrehozott SQL lekérdező nyelv érte el, amely az iparágban egy de facto szabványként terjedt el. Ekkor született meg a \emph{relációs adatbázis-kezelő rendszer} kifejezés is.
\\
Az 1970-es évek után a mai napig, az adatbázisok kifejezés általában relációs rendszerrel asszociálható. Természetesen, a relációs rendszerek mellett megjelentek más adatbázis rendszerek, melyek saját lekérdezőnyelvet használtak, azonban kiderült, hogy ezek sebessége és használhatósága nem veszi fel a versenyt a relációs társaikkal.

